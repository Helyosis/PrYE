\documentclass[12pt, algo]{cours}

\usepackage{pgf-umlcd}

\title{\textbf{\textsc{Ergo -- rapport}}}

\author{Delay Emmanuel-- Desforêts Nicolas}

\makeatletter
\renewcommand{\maketitle}{%
  \thispagestyle{plain}
  \begin{center}%
  \let \footnote \thanks
    {\LARGE \@title \par}%
    \vskip 1.5em%
    {\large \@author}%
  \end{center}%
  \par
  \vskip 1.5em}
\makeatother


\begin{document}

\maketitle

\tableofcontents

\section{Présentation du projet}

\subsection{Présentation du jeu}

Le point de départ est le jeu \href{https://www.catalystgamelabs.com/ergo/}{Ergo}, \og The Game of Proving You Exist!\fg. Les règles détaillées sont \href{https://www.catalystgamelabs.com/download/Ergo%20Rules%202015.pdf}{ici}.

Le jeu est composé de 55 cartes : 4 de chaque variable (A, B, C ou D), 4 de chaque opérateur (ET, OU, $\implies$), 6 cartes NON, 8 parenthèses, 3 cartes Ergo et 10 cartes particulières.

Chaque joueur (4 maximum) se voit assigné une variable (A, B, C ou D) au début du jeu. À chaque manche, les joueurs essaient collectivement de créer une preuve de leur existence tout en réfutant l'existence des autres joueurs. Au début de la manche, chaque joueur reçoit 5 cartes puis, à chaque tour, un joueur pioche deux cartes et doit jouer deux cartes (éventuellement les défausser). Lorsque une carte Ergo est jouée ou qu'il n'y a plus de carte dans la pioche la preuve est terminée. À condition qu'il n'y ait pas de paradoxe, chaque joueur dont l'existence est prouvée reçoit un nombre de points égal au nombre de cartes dans la preuve. Toutes les cartes sont ensuite mélangées et une nouvelle manche est lancée. Le premier joueur ayant 50 points gagne.

Concernant la construction de la preuve, un certain nombre de règles doivent être respectées~:
\begin{itemize}
\item la preuve doit avoir au maximum 4 lignes. Dès qu'elle atteint 4 lignes, toutes les cartes supplémentaires doivent être jouées sur une de ces lignes~;
\item chaque ligne doit être syntaxiquement correcte (deux opérateurs ou deux variables ne peuvent pas se suivre, chaque parenthèse ouvrante doit correspondre à une parenthèse fermante, \dots)~;
\item une carte peut être insérée entre deux cartes déjà posées à condition que le résultat reste syntaxiquement correct.
\end{itemize}

\subsection{Le projet}

Le but est de réaliser une implémentation en Python de ce jeu. Pour cela, plusieurs points sont à traiter, plus ou moins par ordre de difficulté croissante~:
\nopagebreak
\begin{itemize}
\item analyser une ligne de la preuve pour vérifier qu'elle est syntaxiquement correcte~;
\item coder une ligne syntaxiquement correcte sous une forme exploitable (arbre, forme conjonctive normale, forme disjonctive normale, \dots ?)~;
\item déterminer à partir du codage des 4 lignes quelles variables sont prouvées ou s'il y a une contradiction~;
\item réaliser une interface graphique (à priori avec tkinter)~;
\item implémenter une fonction pour pouvoir jouer contre l'ordinateur.
\end{itemize}

Si on finit tout ça et qu'on a peur de s'ennuyer, on pourra toujours creuser pour améliorer la façon dont l'ordinateur joue. Au pire, on demandera à Frédéric Muller de nous prêter ses TetrisBot pour qu'ils apprennent à jouer à Ergo ;-)


\section{Organisation du travail}

\subsection{Outils utilisés}

Nous avons configuré un Rasberry Pi comme serveur pour installer redmine dessus, en nous aidant beaucoup du \href{http://juramaths.fr/redmine/projects/serveur-web-sur-un-raspberry-pi/wiki}{wiki} de Frédéric Muller (merci à lui) et des article de Linux Pratique mis à notre disposition par The Big Boss (loué soit-il).

Nous avons aussi créé un dépôt sur github : \url{https://github.com/isnpaulconstans/Ergo}

La documentation technique est générée avec Sphinx, encore grâce aux articles de GNU/Linux Magasine que notre Big Boss a eu la bonté de nous fournir (Il n'en sera jamais assez remercié \footnote{Le cirage de pompe peut-il augmenter significativement la note de ce module ?}).

Les résultats sont disponibles sur \url{http://paulconstans.ddns.info/redmine/projects/ergo} et sur \url{http://paulconstans.ddns.info/documentation/}.

\subsection{Répartition du travail}

Après quelques discutions, le travail s'est assez naturellement réparti. Emmanuel Delay s'est chargé de la partie algorithmique tandis que Nicolas Desforêts s'est occupé de l'interface graphique. Comme nous travaillons tous les deux dans le même lycée, nous avons pu nous voir régulièrement pour faire la jointure entre nos deux parties et nous mettre d'accord sur les étapes suivantes.

\section{Solutions techniques}

Nous utilisons Tkinter pour l'interface graphique. Avec en complément messagebox.

Il a fallu créer les images des cartes, en choisissant une dimension pratique pour la gestion du placement de cartes. La première dimension était trop importante et ne permettait pas de faire des lignes de preuve suffisamment longues. Il a juste fallu redimensionner nos images, car nous avions prévus des constantes (CARD\_WIDTH et CARD\_HEIGHT) en cas de changement.

Nous avons choisi de créer nos images au format gif. Une carte particulière, "back" associée à l'image carteDos.gif permet d'afficher les mains des autres joueurs face cachée.

Les cartes sont gérées à l'aide du dictionnaire IMAGE.Nous avons eu des problèmes au début avec nos images, Tkinter ne reconnaissant pas celles-ci.

En cas de sortie prématurée du programme, il arrive (principalement avec Pyscripter) que nous ayons encore des soucis : tkinter ne retrouve plus les images à afficher. Relancer Pyscripter permet de palier au problème. Nous aimerions comprendre exactement les raisons de ce bug.

\smallskip

Notre classe ErgoGui gère toute la partie graphique du jeu.

La gestion du jeu se fait à la souris. On peut attraper (bouton gauche), déplacer (bouton gauche maintenu) et déposer (bouton gauche relâché) la carte à l'aide des méthodes select(), move() et drop().

Le bouton droit permet, associé à la méthode switch, si la carte est une parenthèse, de la retourner.

Il y a eu un problème de définition du deck : quand nous avons réalisé la méthode permettant de retourner les parenthèses, la modification s'appliquait à toutes les parenthèses du même type. Une liste en compréhension a résolu le problème.

Les règles du jeux sont dans un fichier texte. Lors de l'appel de la méthode nous faisons une lecture du fichier puis un affichage dans une fenêtre messagebox.


\section{Algorithmes utilisés}

\subsection{Passage en notation polonaise inversée : algorithme Shunting-yard}

Une des premier problème algorithmique a été de transformer l'écriture algébrique des preuves en une notation plus utilisable. Ayant pas mal travaillé avec mes élèves sur l'évaluation d'une expression en notation polonaise inversée (NPI), je me disais que je devrais arriver à quelque chose si je pouvais transformer l'écriture algébrique en NPI. J'ai fait quelques recherches la dessus, et je suis tombé sur l'algorithme de \href{https://fr.wikipedia.org/wiki/Algorithme_Shunting-yard}{Shunting-yard}.

Je l'ai légèrement adapté au contexte (proposition logique au lieu de d'expression mathématique) pour obtenir l'algorithme \ref{Shunting-yard}.


\begin{algorithm}
\caption{Algorithme de passage en notation polonaise inversée}
\label{Shunting-yard}
\Entree{Une liste \texttt{input} de cartes (propositions ou connecteurs)}
\Sortie{Une liste npi correspondant a la notation polonaise inversée de l'entrée}
\Trait{
Créer une \texttt{pile} vide\;
\texttt{npi} $\leftarrow []$ \;
\PourCh{\texttt{carte} de \texttt{input}}{
	\uSi{\texttt{carte} est une lettre}{ajouter \texttt{carte} à \texttt{npi}}
	\uSinonSi{\texttt{carte} est un parenthèse ouvrante}{empiler \texttt{carte}}
	\uSinonSi{\texttt{carte} est une parenthèse fermante}{
		\Tq{pile est non vide et que le sommet de la pile n'est pas une parenthèse ouvrante}{dépiler une carte et l'ajouter à \texttt{npi}}
		\eSi{pile est vide}{quitter \tcp*[l]{Problème de parenthésage}}{dépiler la parenthèse ouvrante}
		}
	\Sinon{
		\Tq{pile est non vide et que le sommet de la pile a une priorité supérieure à \texttt{carte}}{dépiler une carte et l'ajouter à \texttt{npi}}
		empiler \texttt{carte}
		}
	}
\Tq{pile est non vide}{
	dépiler une carte et l'ajouter à \texttt{npi}\;
	\Si{la carte est une parenthèse ouvrante}{quitter \tcp*[l]{Problème de parenthésage}}
	}
}
\end{algorithm}

\subsection{Évaluation de la preuve}

\subsubsection{Force brute}

Ici, mon idée a été d'attaquer le problème en force brute : tester tous les modèles possible (comme il y a 4 variables, il y a seulement $2^4=16$ possibilités) et pour chacun évaluer la preuve. Si le résultat est Vrai, c'est que le modèle est admissible et je le mémorise. Ensuite, je regarde pour chaque variable si elle a toujours la même valeur (Vrai ou Faux) dans tous les modèles admissibles. Si c'est la cas, la variable est prouvée ou niée.

\medskip
D'après le cours qu'on a eu pour l'instant sur la logique avec Line Jakubie-Jamet, cette méthode constitue une preuve sémantique, mais c'est équivalent à une preuve syntaxique.

\medskip

Pour l'évaluation d'une formule, j'ai utilisé l'algorithme classique d'évaluation d'une expression en NPI (algorithme \ref{evalNPI}).

\begin{algorithm}
\caption{Algorithme d'évaluation d'une formule}
\label{evalNPI}
\Entrees{Une liste \texttt{npi} de carte en NPI et une interprétation}
\Sortie{L'évaluation de la liste}
\Trait{
Créer une pile vide\;
\PourCh{\texttt{carte} de \texttt{npi}}{
	\uSi{\texttt{carte} est une lettre}{empiler sa valeur dans l'interprétation}
	\uSinonSi{\texttt{carte} est un opérateur binaire}{
		dépiler les deux dernières valeurs\;
		effectuer l'opération entre ces valeurs\;
		empiler le résultat
		}
	\Sinon(\tcp*[h]{c'est un opérateur unaire, le NON}){
		dépiler la dernière valeur\;
		empiler sa négation
		}
	}
\Retour{le sommet de \texttt{pile}} \tcp*[l]{qui ne doit avoir qu'un élément}
}
\end{algorithm}

\subsubsection{Algorithme de Davis-Putnam-Logemann-Loveland (DPLL)}

Même si l'algorithme précédent marche bien dans le contexte qui nous intéresse (4 variables propositionnelles), comme l'algorithme DPLL (algorithme \ref{algoDPLL}) nous a été présenté dans le cours de logique, j'ai voulu l'implémenter. Pour cela, il fallait commencer par transformer la preuve au format NPI en une Forme Conjonctive Normale (FCN) sous forme d'une liste de clauses. Cela se fait en 4 étapes :
\begin{itemize}
\item élimination des implications ($A \implies B \equiv \neg A \vee B$) ;
\item utilisation des lois de Morgan ($\neg (A \vee B) \equiv \neg A \wedge \neg B$ et $\neg (A \wedge B) \equiv \neg A \vee \neg B$) ;
\item élimination des doubles négations ($\neg \neg A \equiv A$) ;
\item développement ($A \vee (B \wedge C) \equiv (A \vee B) \wedge (A \vee C)$).
\end{itemize}

\begin{algorithm}
\caption{Algorithme de Davis-Putnam-Logemann-Loveland (DPLL)}
\label{algoDPLL}
\Entrees{Une liste de clauses \texttt{clause\_list} et un modèle partiel \texttt{model}}
\Sortie{Vrai si on peut compléter le modèle partiel en un modèle complet, Faux sinon}
\Trait{
	\Tq{\texttt{clause\_list} contient une clause unitaire \texttt{[lit]}}{
		ajouter la valeur de \texttt{lit} au modèle \;
		supprimer toutes les clauses contenant \texttt{lit} \;
		supprimer $\neg \texttt{lit}$ de toutes les clauses où il apparaît \;
	}
	\lSi{il ne reste plus de clause}{\Retour{Vrai}}
	\lSi{\texttt{clause\_list} contient une clause vide}{\Retour{Faux}}
	\PourCh{variable \texttt{var} non testée}{
		$\texttt{model\_tmp}[\texttt{var}] \leftarrow $ Faux (resp. Vrai)\; 
		\Si(\tcp*[f]{La négation est prouvée}){non DPLL(\texttt{clause\_list}, \texttt{model\_tmp})}{
			$\texttt{model}[\texttt{var}] \leftarrow$ Vrai (resp. Faux) \;
			supprimer toutes les clauses contenant $\neg \texttt{var}$ \;
			supprimer \texttt{var} de toutes les clauses où il apparaît \;
			\Retour{DPLL(\texttt{clause\_list}, \texttt{model})}
		}
	$\texttt{model}[\texttt{var}] \leftarrow None$ \tcp*{\texttt{var} est indécidable}
	}
	\Retour{Vrai} \tcp*{Toutes les variables on été testées}
}

\end{algorithm}

\subsection{Jeu de l'ordinateur}

Une première étape consiste, si on a au moins deux parenthèses en main, à se débrouiller pour en avoir au moins une ouvrante et une fermante.

Ensuite, on détermine l'ensemble des coups possibles en essayant de poser chacune des cartes de la main à un endroit de la preuve.

Pour l'instant, l'ordinateur choisit alors un coup au hasard parmi l'ensemble des coups possibles.

\section{Évolutions à venir}

\begin{itemize}
\item Gérer les scores, et la fin de partie.
\item Proposer au(x) joueur(s) de rentrer leurs noms.
\item Pas mal de factorisations possibles dans ErgoGui (séparer le canvas dans une classe ErgoCanvas dédiée, fonctions de calcul pour passer des coordonnées dans le canvas à un numéro de prémisse et un numéro de colonne, \dots)
\item Gérer les cartes spéciales (Ergo, Tabula Rasa, \dots)
\item Tester d'autres méthodes pour le jeu de l'ordinateur. Il faudrait en particulier attribuer un score à chaque coup en fonction de ce qui est prouvé. Ensuite, j'avais éventuellement pensé à un minimax en testant toutes les cartes qui n'ont pas encore été jouées, mais je pense que ça va faire trop de calculs. Peut-être qu'en faisant quelques parties, une stratégie se présentera\dots Sinon, il restera la solution de faire appel à un ami ?
\end{itemize}

\appendix

\section{Diagramme des classes}

\input{"diagramme des classes"}
\end{document}

